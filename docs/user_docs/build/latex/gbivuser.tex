%% Generated by Sphinx.
\def\sphinxdocclass{report}
\documentclass[letterpaper,10pt,english,openany,oneside]{sphinxmanual}
\ifdefined\pdfpxdimen
   \let\sphinxpxdimen\pdfpxdimen\else\newdimen\sphinxpxdimen
\fi \sphinxpxdimen=.75bp\relax
\ifdefined\pdfimageresolution
    \pdfimageresolution= \numexpr \dimexpr1in\relax/\sphinxpxdimen\relax
\fi
%% let collapsible pdf bookmarks panel have high depth per default
\PassOptionsToPackage{bookmarksdepth=5}{hyperref}

\PassOptionsToPackage{warn}{textcomp}
\usepackage[utf8]{inputenc}
\ifdefined\DeclareUnicodeCharacter
% support both utf8 and utf8x syntaxes
  \ifdefined\DeclareUnicodeCharacterAsOptional
    \def\sphinxDUC#1{\DeclareUnicodeCharacter{"#1}}
  \else
    \let\sphinxDUC\DeclareUnicodeCharacter
  \fi
  \sphinxDUC{00A0}{\nobreakspace}
  \sphinxDUC{2500}{\sphinxunichar{2500}}
  \sphinxDUC{2502}{\sphinxunichar{2502}}
  \sphinxDUC{2514}{\sphinxunichar{2514}}
  \sphinxDUC{251C}{\sphinxunichar{251C}}
  \sphinxDUC{2572}{\textbackslash}
\fi
\usepackage{cmap}
\usepackage[T1]{fontenc}
\usepackage{amsmath,amssymb,amstext}
\usepackage{babel}



\usepackage{tgtermes}
\usepackage{tgheros}
\renewcommand{\ttdefault}{txtt}



\usepackage[Bjarne]{fncychap}
\usepackage[,numfigreset=1,mathnumfig]{sphinx}

\fvset{fontsize=auto}
\usepackage{geometry}


% Include hyperref last.
\usepackage{hyperref}
% Fix anchor placement for figures with captions.
\usepackage{hypcap}% it must be loaded after hyperref.
% Set up styles of URL: it should be placed after hyperref.
\urlstyle{same}

\addto\captionsenglish{\renewcommand{\contentsname}{Contents:}}

\usepackage{sphinxmessages}
\setcounter{tocdepth}{3}
\setcounter{secnumdepth}{3}


\title{Gbiv User Documentation}
\date{Dec 02, 2022}
\release{1.1}
\author{Dux D\sphinxhyphen{}zine}
\newcommand{\sphinxlogo}{\vbox{}}
\renewcommand{\releasename}{Release}
\makeindex
\begin{document}

\ifdefined\shorthandoff
  \ifnum\catcode`\=\string=\active\shorthandoff{=}\fi
  \ifnum\catcode`\"=\active\shorthandoff{"}\fi
\fi

\pagestyle{empty}
\sphinxmaketitle
\pagestyle{plain}
\sphinxtableofcontents
\pagestyle{normal}
\phantomsection\label{\detokenize{index::doc}}


\sphinxstepscope


\chapter{Getting Started}
\label{\detokenize{getting_started:getting-started}}\label{\detokenize{getting_started:id1}}\label{\detokenize{getting_started::doc}}
\sphinxAtStartPar
Welcome to Gbiv’s user documentation! Here you will find information that will help to guide you through using our web application and give you some background on our recommendation system. For questions about this document or anything else related to Gbiv, feel free to get in touch with the DUX D\sphinxhyphen{}Zine team at \sphinxhref{mailto:duxdzine@gbiv.com}{duxdzine@gbiv.com} or by calling our help line at 1 (800) 867\sphinxhyphen{}5309.

\sphinxAtStartPar
Below we go through each section of the user documentation and give a brief description of what content is available there.


\section{“Overview” Section}
\label{\detokenize{getting_started:overview-section}}
\sphinxAtStartPar
In this section we give an overview of the functionality of Gbiv and discuss some of the ways you can apply the power of Gbiv to your everyday life.


\section{“The Site” Section}
\label{\detokenize{getting_started:the-site-section}}
\sphinxAtStartPar
This section of this document describes the website that acts as your access point to the powerful tools that our application provides. Here you will find descriptions of the individual pages, tutorials on how to interact with the user interface, and general site navigation.


\section{“Format of Inputs and Outputs” Section}
\label{\detokenize{getting_started:format-of-inputs-and-outputs-section}}
\sphinxAtStartPar
In this section we give detailed descriptions of the format requirements for photos uploaded to Gbiv as well as helpful descriptions for the palettes and colors that Gbiv gives back to users.


\section{“Walkthrough” Section}
\label{\detokenize{getting_started:walkthrough-section}}
\sphinxAtStartPar
In this section we provide a basic sequence of steps to follow if you would like to get familiar with the Gbiv. This is a good place to start if you want to jump right into using the application.


\section{“Color Theory Basics” Section}
\label{\detokenize{getting_started:color-theory-basics-section}}
\sphinxAtStartPar
Here we give a brief introduction to the core principles of color theory to better understand the science behind Gbiv. We then show how these principles are applied when you upload your photos. This section will give you further insight if you are wishing to learn more about how the palette and color recommender system works.

\sphinxstepscope


\chapter{Overview of The Application}
\label{\detokenize{overview:overview-of-the-application}}\label{\detokenize{overview:overview}}\label{\detokenize{overview::doc}}
\sphinxAtStartPar
Gbiv is a web application created for the designer in all of us. Design is everywhere\textendash{}from the clothes you put on in the morning to the room you wake up in\textendash{}and an essential part of design is color. Our application provides a powerful tool for utilizing color theory when making design decisions.

\sphinxAtStartPar
In the digital space there are color classification systems structured around color theory principles, but in the physical world we often lack such convenience. Gbiv bridges this gap by allowing users to upload images of physical objects and extract the primary color of what is pictured. It then goes a step further by performing automated design analysis and suggesting related colors and palettes. This lets everyone, whether they are professionals or beginners, have access to the power of color theory in their everyday life.

\sphinxAtStartPar
Have you ever been painting your house and needed to match a new color to the existing paint? Do you ever find yourself wondering what color of pants match with your new sweater? Or maybe you are at the furniture store and can’t figure out if a couch would “fit” with your living room carpet? These are all examples of situations where Gbiv can save the day.

\sphinxAtStartPar
Beyond this primary functionality, Gbiv offers several other services to support end\sphinxhyphen{}users. This includes providing example palettes for design inspiration and giving information about color theory principles so that users can understand the science behind Gbiv’s recommendations.

\sphinxstepscope


\chapter{The Site}
\label{\detokenize{site:the-site}}\label{\detokenize{site:site}}\label{\detokenize{site::doc}}
\sphinxAtStartPar
The website at which users can access Gbiv is divided into four pages which all link to each other through a shared navbar at the top of each page. When the user first visits the URL, they will find themselves on the “Main Page” (see \hyperref[\detokenize{site:main-pg}]{Section \ref{\detokenize{site:main-pg}}} for more details). From there, they can simply click the title of the page they wish to go to from the list in the top left corner.


\section{Main Page}
\label{\detokenize{site:main-page}}\label{\detokenize{site:main-pg}}
\sphinxAtStartPar
Here is where users will start when the visit Gbiv. This page provides the interface to use the application’s primary function which is uploading an image for color analysis and suggestions. At the top of the main page you can see bold text prompting you to upload an image. For more information about how to format image inputs see \hyperref[\detokenize{input_formatting:input}]{Section \ref{\detokenize{input_formatting:input}}}.

\sphinxAtStartPar
Once the user has uploaded an image, there will be a short period in which Gbiv will analyze the colors. After analysis is complete, users will be taken to a new page where their image along with suggested palettes are displayed.  Continue scrolling down past the palettes and you will find several individual related colors that are separated into different categories.


\section{Example Palettes Page}
\label{\detokenize{site:example-palettes-page}}\label{\detokenize{site:ex-palettes-pg}}
\sphinxAtStartPar
On this page several example palettes are shown with the intention of inspiring design decisions and providing pre\sphinxhyphen{}made color schemes for those without a place to start.

\sphinxAtStartPar
The palettes are displayed visually in the form of blocks divided into 4 different colors. By hovering over these blocks you can find the hex code for each color in the set.


\section{Color Theory Page}
\label{\detokenize{site:color-theory-page}}\label{\detokenize{site:color-theory-pg}}
\sphinxAtStartPar
If you are wanting to learn more about color theory this is the place for you! Here you will find a brief introduction to the subject and further details on the principles used in creating Gbiv’s recommender system.


\section{About Us Page}
\label{\detokenize{site:about-us-page}}\label{\detokenize{site:about-us-pg}}
\sphinxAtStartPar
Last but not least, the “About Us” page gives some background on the DUX D\sphinxhyphen{}Zine team and the Gbiv project in particular. If you wish to get in touch with us or have any questions about the application, you can also find contact information here.

\sphinxstepscope


\chapter{Format of Inputs and Outputs}
\label{\detokenize{input_formatting:format-of-inputs-and-outputs}}\label{\detokenize{input_formatting:input}}\label{\detokenize{input_formatting::doc}}
\sphinxAtStartPar
The photos you upload to Gbiv must be in the standardized image formats of .jpg, .jpeg, or .png so that they can be interpreted by the system.

\sphinxAtStartPar
The output of Gbiv is the colors you receive after uploading. When using the application, you will be able to see the colors visually, but also be able to get the hex codes that correspond to the colors displayed. Hex color codes are a way of classifying the colors used on the web. They come in the form of a 6 character code with a pound symbol (\#) in front of it. This system of defining colors is used extensively in web development, but is also recognized in a variety of other domains such as commercial paint and printing.

\sphinxstepscope


\chapter{Walkthrough}
\label{\detokenize{walkthrough:walkthrough}}\label{\detokenize{walkthrough:id1}}\label{\detokenize{walkthrough::doc}}
\sphinxAtStartPar
If you are a first time user to Gbiv or need a little refresher, below we have provided a small tutorial to get you more acquainted with the application.

\sphinxAtStartPar
To begin this tutorial, simply visit Gbiv on the web at the URL (\sphinxhref{https://skumperdump.pythonanywhere.com/)--please}{https://skumperdump.pythonanywhere.com/)\textendash{}please} excuse the URL name haha. Gbiv is compatible with all major web browsers.


\section{Step 1: Navigating the Page}
\label{\detokenize{walkthrough:step-1-navigating-the-page}}
\sphinxAtStartPar
Once you enter the URL for Gbiv into your browser, you will be brought to the main page which also acts as the upload page for Gbiv’s primary functionality. At the top of your browser you will see a lavender navigation bar that has Gbiv’s logo as well as a list of all pages on the site. You can visit any of these pages by clicking the name displayed at the top. For more information on what each section offers you can visit the “The Site” section of this user documentation (\hyperref[\detokenize{site:site}]{Section \ref{\detokenize{site:site}}}).


\section{Step 2: Selecting an Image}
\label{\detokenize{walkthrough:step-2-selecting-an-image}}
\sphinxAtStartPar
To access the main functionality of Gbiv, you will need an image to upload that is formatted as a .png, .jpg, or .jpeg (as mentioned in \hyperref[\detokenize{input_formatting:input}]{Section \ref{\detokenize{input_formatting:input}}}) and saved somewhere on your local machine.

\sphinxAtStartPar
To begin the upload process simply click on the button that says “Choose File” on the main page. After this, you will be taken to a file explorer window specific to your operating system which will allow you to navigate to wherever your image is stored. Once you have selected the image you wish to upload, click “Open” to close the file selector tab and return to the website (NOTE: the exact text for the button to select the file may not be “Open” depending on the OS you are using).

\sphinxAtStartPar
Now the name of the image will be displayed next to the “Choose File” button. If you wish to change the image you have selected, simply click the “Choose File” button and repeat the process described above.


\section{Step 3: Uploading the Image}
\label{\detokenize{walkthrough:step-3-uploading-the-image}}
\sphinxAtStartPar
Once you have selected the image you wish to input into Gbiv, the process of uploading is very straightforward. Simply click the button to the right of the your selected file name which reads “Upload.” After pressing this button, the color analysis process will begin\textendash{}please allow up to 20 seconds for this computation to complete. If it is taking too long, try resizing your photo and re\sphinxhyphen{}uploading\textendash{}the smaller the file size, the faster the process will be.


\section{Step 4: Viewing the Suggested Colors}
\label{\detokenize{walkthrough:step-4-viewing-the-suggested-colors}}
\sphinxAtStartPar
After the color extraction and analysis computations have been completed, you will be re\sphinxhyphen{}directed to a page that displays your suggested palettes and related colors. At the top of the page, you can see the image you uploaded and below that there will be several categories of palettes and related colors.

\sphinxAtStartPar
Each palette is displayed as a rectangle that is divided into 4 horizontal bars that represent the 4 individual colors in the palette. To the right of these bars is the hex code that digitally represents the color shown.

\sphinxAtStartPar
The related colors are shown below the palettes. They are displayed as squares filled in with a single color and are also grouped logically based on their relation to the dominant color from the image.


\section{Step 5: Exploring Further}
\label{\detokenize{walkthrough:step-5-exploring-further}}
\sphinxAtStartPar
If you would like to know more about the palettes and colors that have been recommended based off of the photo you uploaded, go over to the color theory page for a basic introduction to the science behind our color analysis.

\sphinxAtStartPar
If you are looking for palette inspiration, but don’t have a specific photo to match with in mind: visit the “Sample Palettes” page to view some palettes that have been generated by Gbiv in the past.

\sphinxAtStartPar
If you would like to get in touch with the team behind Gbiv or are interested in the story behind our project, you can find what you need in the “About Us” page.

\sphinxstepscope


\chapter{Color Theory Basics}
\label{\detokenize{color_theory:color-theory-basics}}\label{\detokenize{color_theory:color-theory}}\label{\detokenize{color_theory::doc}}
\sphinxAtStartPar
\sphinxstylestrong{What is Color Theory?}

\sphinxAtStartPar
Color theory is the thought process behind putting colors together for a design based on the color wheel. It explains how we as humans view color and the effects of mixing colors together. That being said, color theory is relevant in many fields that involve visual communication such as marketing,art, graphic design, and etc. A very simple example would be putting light colored text over a dark background to make things legible for the user or reader.

\sphinxAtStartPar
“Color! What a deep and mysterious language, the language of dreams.”
— Paul Gauguin, Famous post\sphinxhyphen{}Impressionist painter

\sphinxAtStartPar
\sphinxstylestrong{Isaac Newton and Color Theory}

\sphinxAtStartPar
Sir Isaac Newton played a significant role in creating color theory. He put colors into three groups: primary, secondary and tertiary. Primary colors are red, blue and yellow. Secondary colors are those produced when mixing primary colors together. An example would be green which is made by blue and yellow. Lastly, tertiary is the mix of the primary and secondary colors. Examples would be blue\sphinxhyphen{}green, blue\sphinxhyphen{}violet, red\sphinxhyphen{}orange, etc.

\sphinxAtStartPar
\sphinxstylestrong{Color Properties}

\sphinxAtStartPar
There are three important properties that colors have. The first one is hue, which is how the color appears. The second one is chroma which indicates if there are any shades or tints added. Lastly, there is lighting, which indicated the saturation of the color.

\sphinxAtStartPar
Knowing about the different groups of colors, how they mix, and the properties of colors are an essential part of UX design, art and marketing.

\sphinxAtStartPar
\sphinxstylestrong{How Gbiv Uses Color Theory}

\sphinxAtStartPar
Our program utilizes color theory by substituting/changing different HSL values. HSL stands for Hue, Saturation, and Lightness. The HSL code takes in 3 parameters. The first is how many degrees around the wheel a particular color is. The next describes how concentrated we want it to be. The last parameter is how bright the color should be ( 50 \%: Balanced between black and white). Shifting the hue values on the color wheel will change the base color. For example, if we upload an image with a dominant color of red (0 degrees), then we add .5 (hue), this means our program will output the color cyan (180 degrees).



\renewcommand{\indexname}{Index}
\printindex
\end{document}